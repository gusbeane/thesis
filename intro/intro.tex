% !TEX root = ../ms.tex

% \textit{\hspace{24pt}\\ 
% \hspace{92pt}-Siddhartha Gautama}

\begin{adjustwidth}{.8cm}{0cm}
\textit{Do not believe in anything simply because you have heard it. Do not believe in anything simply because it is spoken and rumored by many. Do not believe in anything simply because it is found written in your religious books. Do not believe in anything merely on the authority of your teachers and elders. Do not believe in traditions because they have been handed down for many generations. But after observation and analysis, when you find that anything agrees with reason and is conducive to the good and benefit of one and all, then accept it and live up to it.}

\hspace{9cm} -- Siddhartha Gautama
\end{adjustwidth}

\section{Overview}
\label{sec:overview_MW}

\subsection{Why the Milky Way?}
By virtue of being embedded within the Milky Way \citep{1610snml.book.....G}, we have a unique perspective on its formation and evolution. Crucially, our ability to observe it on a star-by-star basis has led to an exquisitely detailed understanding. We can decipher events in its past that we have no chance of replicating for any external galaxies in the near future. 

However, this vantage point that provides us with an intimate view of our galaxy's internals also obscures some of its most basic features. The first barred galaxy was observed over 235 years ago, well before it could be interpreted \citep{1789RSPT...79..212H}. The idea that the Milky Way might have a bar was not seriously considered until 175 years later \citep{1964IAUS...20..195D} and was not definitively discovered for another 27 years \citep{1991ApJ...379..631B}. Studying the Milky Way would be like studying the Sun from its interior, blind to its surface, and then trying to compare it to other stars.\footnote{Other complications may arise.}

This discrepancy between our vantage point of the Milky Way and that of other galaxies complicates the transfer of knowledge of one to the other. Nonetheless, demanding that our understanding of the Milky Way is \textit{consistent} with that of external galaxies can lead to useful insights into either. For example, in Chapter~\ref{ch:gasbar}, we will discuss how the Milky Way's bar rotates quickly, much in line with the rotation rates of bars in exteranl galaxies. And in Chapters~\ref{ch:GSEgas} and~\ref{ch:Mgdec}, we will discuss the possibility that a quiescent phase in the Milky Way's past led to the emergence of the $\alpha$-bimodality, a proposal that aligns with quiescent galaxies at high-$z$ now routinely observed by \textit{JWST}.

And above all else, the Milky Way is an interesting object of study because it is our home.

\subsection{Structural Properties}\label{intro:ssec:struct_prop}
It has become common practice to decompose the Milky Way into distinct structural components: a dark matter halo, a stellar bulge, thin and thick stellar disks, a gas disk, and the gaseous circumgalactic medium \citep[for an extensive review, see][]{2016ARA&A..54..529B}.\footnote{The gas phases are often further divided into cold and hot, or ionized and neutral, components.}

The Milky Way's dark matter halo has a mass of $\sim1$--$2\times10^{12}\,\Msun$ \citep[Table 8 in][]{2016ARA&A..54..529B}. Halos at this mass are the most efficient at producing stellar material, with a stellar to halo mass fraction of $\sim2$--$3\%$ \citep{2013ApJ...770...57B}. It is commonly understood that halos less massive than $\sim10^{12}\,\Msun$ have their star formation inhibited by stellar feedback (e.g., stellar winds and supernovae) while more massive halos are inhibited by feedback from active galactic nuclei (AGN) 

In the stellar mass-color plane, most galaxies lie either in a star-forming blue sequence or a quenched red sequence \citep{2006MNRAS.373..469B}. The Milky Way appears to lie in the relatively underpopulated ``green valley'' \citep{2011ApJ...736...84M}, implying a modest star formation rate (SFR) of $\sim1.7\,\Msunyr$ \citep{2015ApJ...806...96L}. Like the vast majority of galaxies, the Milky Way hosts a supermassive black hole (SMBH) at its center (Sgr~A$^{\star}$) which has a mass very precisely measured to $4.3\times10^{6}\,\Msun$. However, there is a strong correlation between a galaxy's bulge mass and its central black hole mass, and Sgr~A$^{\star}$ lies a factor of $\sim5$--$6$ below this relation \citep{2013ARA&A..51..511K}.

A more detailed summary of its structural properties can be found in \citet{2016ARA&A..54..529B}.

\subsection{Observations}
This thesis will mostly concern stellar observations, for which there are three main properties that can be measured. In order of increasing difficulty to measure, these are dynamics (position and velocity), chemical abundances, and age.

\subsubsection{Dynamics}
For dynamics, radial velocities can be measured relatively easily from the redshifting of certain spectral lines, which can be done from the ground. The other five components are inferred from highly precise measurements of the on-sky position of a source over a long period of time. Using the parallax effect, one can simultaneously fit for the distance of a source as well as its apparent motion on the sky (proper motion), which can be converted to a 3D velocity when combined with the radial velocity and distance measurement. The \textit{Gaia} mission, the successor to the \textit{Hipparcos} mission \citep{1997ESASP1200.....E}, has used this technique to measure the six-dimensional positions and velocities of $\sim1.46$~billion sources \citep{2023A&A...674A..37G}.

\subsubsection{Chemistry}
Much has been learned about the history of the Galaxy from precise measurements of stellar positions and velocity. However, this inference suffers from an inherent difficulty in needing to know the potential of the Galaxy, which is uncertain even just at present. On the other hand, surface abundances -- the precise ratio of different chemical elements on the surfaces of stars -- do not change over time.\footnote{With some exceptions, e.g., drudge up, binary mass transfer, and disk accretion. However, for most elements in most stars, surface abundances do not change.} Therefore, the surface abundances of stars reflect the composition of the interstellar medium (ISM) at the time of their formation, and precise measurements of these abundances gives a history of the composition of the Galaxy. Unlike in dynamics, the observational campaigns to observe these abundances is much more heterogeneous, with surveys including SAGES \citep{2023ApJS..268....9F}, GALAH \citep{2015MNRAS.449.2604D}, H3 \citep{2019ApJ...883..107C}, LAMOST \citep{2012RAA....12..723Z}, and APOGEE, a part of the SDSS surveys \citep{2017AJ....154...94M}. The last of these, APOGEE, has measured the abundances of over 650,000 stars in its most recent data release based on SDSS-IV, with millions more to come with SDSS-V \citep{2017arXiv171103234K}.

\subsubsection{Ages}
Because we do not \textit{a priori} know when and where a given star formed, interpreting the record encoded in the chemical abundances of stars is far from straightforward. Precisely determining where a given star formed is impossible from present-day measurements, but inferring a star's age from its properties is, in principle, possible. The prospects for measuring ages of old field stars (i.e., not in clusters) is bleak. Unfortunately, it is precisely these stars which this thesis is chiefly concerned with. Stellar ages can be inferred from the decay of radioactive elements, but this can only be done on very metal poor stars where such lines can be observed \citep[e.g.][]{2010A&A...509A..84L}. Ages can be inferred from the slowdown of stellar rotation due to magnetic braking, but this is difficult for older stars because the slowdown becomes increasingly gradual with age \citep[e.g.][]{2020AJ....160...90A}.

The most promising approach currently is through astroseismology. Here, the masses of evolved stars (e.g., red giant or red clump) is inferred from measurements of resonant oscillations -- in particular, the frequency of peak amplitude and the frequency spacing between modes. This technique does become more difficult towards lower metallicity, yet it remains tractable. A total of $\sim9000$ stars have relatively precise ($\sigma_{\textrm{age}}/\textrm{age} < 0.25$) ages \citep{2024arXiv241000102P}.

\subsubsection{Key Events}
From these three properties of stars in the Milky Way, two key events in the history of the Milky Way have been inferred. These events are the main interest of this thesis. There is strong evidence that the Milky Way underwent a significant merger with the satellite galaxy termed \textit{Gaia}-Sausage-Enceladus (GSE) $\sim8\,\Gyr$ ago \citep{2018MNRAS.478..611B,2018Natur.563...85H}. This satellite galaxy had a stellar mass of $\sim5\times10^8\,\Msun$ (a total mass ratio of $1:2.5$ with the proto-Milky Way), and approached on a retrograde orbit that radialized after its first pericentric passage \citep{2021ApJ...923...92N}. Nearly all stars on radial orbits in the inner $25\,\kpc$ of the Galaxy are linked to GSE \citep{2020ApJ...901...48N}, and the inner halo is tilted as a result of the merger \citep{2022AJ....164..249H}.

Additionally, the bar is thought to have formed around the same time \citep{2019MNRAS.490.4740B,2024MNRAS.530.2972S}.

\subsection{Simulation Techniques}
We are mostly concerned with the application of simulation techniques to interpreting observational understandings of the Milky Way. We next turn to a brief overview of the specific techniques used in this thesis. Because our main focus is not on these techniques themselves but rather on their application to the Milky Way, we will almost exclusively focus on the techniques used in this work. In particular, we have made use of the \AREPO{} code. The details of this code is documented in the original release paper \citep{2010MNRAS.401..791S}, as well as subsequent work that improved its convergence properties and its public release \citep{2016MNRAS.455.1134P,2020ApJS..248...32W}. Additional information on certain algorithms is available in the \texttt{GADGET} papers \citep{2001NewA....6...79S,2005MNRAS.364.1105S,2021MNRAS.506.2871S}.

\subsubsection{Gravity}
In \AREPO{}, gravity is treated using a \citet{1986Natur.324..446B} oct-tree algorithm. Each particle is assigned to a tree that splits the domain into hierarchical subboxes by splitting any region with more than one particle into eight new nodes. At each level of the hierarchy, the multipole moments and center of mass of each node is stored.\footnote{In \AREPO{} only the monopole moment (i.e., total mass) is used.} The force on a particle is then computed as a particle--node interaction using the multipole moment approximation. Nodes are descended further if they are sufficiently close to a particle to warrant the approximation too inaccurate. This algorithm has a $\mathcal{O}\left(N \log{N}\right)$ scaling, as opposed to the $\mathcal{O}\left(N^2\right)$ for the brute force approach.

Optionally, a particle mesh algorithm can be used for long-range forces in which the mass distribution is binned onto a regular Cartesian grid and the force is computed using a Fourier transform version of the Poisson equation. This is typically used in cosmological simulations.

To avoid spurious heating and prohibitively small timesteps from short-range interactions, gravity is softened at small scales. Each resolution element is assigned a softening length $\epsilon$. Beyond $2.8\epsilon$ the force between two particles is Newtonian, but the force between the two particles is diminished when their separation is $\lesssim\epsilon$.

\subsubsection{Magnetohydrodynamics}
In \AREPO{}, magnetohydrodynamics (MHD) is solved using a second-order, finite-volume method. The fluid is discretized as an unstructured Voronoi mesh, with the primitive variables density, energy density, velocity, and magnetic fields ($\rho$, $u$, $\mathbf{v}$, and $\mathbf{B}$, respectively) at each cell's center stored with the element. The gradient of these quantities within each cell is fit using a least squares estimate with the neighboring cells. These gradients are used for a piece-wise linear reconstruction of the fluid quantities to the center of mass of each cell face. The Riemann problem is then solved at the cell faces in order to compute the flux of each fluid quantity between each cell.

\subsubsection{Radiative Cooling}
Gas elements are typically allowed to lose thermal energy in galaxy formation simulations through atomic metal-line cooling \citep{1992ApJ...399L.109K,1996ApJS..105...19K}. In Illustris~TNG, the elements H, He, C, N, O, Ne, Mg, Si, and Fe are followed. The cooling rate of a specific cell is computed by interpolating from a cooling table which uses the total metallicity of the table \citep{2008MNRAS.385.1443S,2009MNRAS.393...99W}, with an additional heating term from the uniform UV background.

\subsubsection{Time Integration}
Numerous criteria impose timestep constraints on each resolution element, and the element's actual timestep must satisfy the most restrictive constraint. The gravitational constraint scales inversely with a particle's acceleration, limiting the timestep of particles with high accelerations. For hydrodynamics it is determined by the cell's signal speed (incorporating both the sound speed and Alfvén speed). Additional physics, such as star formation, can introduce further constraints \citep[see Section~7.1 in][]{2020ApJS..248...32W}.

Advancing the simulation with a global timestep equal to the smallest timestep of all particles is prohibitively expensive, especially in galaxy simulations where the necessary timesteps can vary by orders of magnitude. To allow for resolution elements to have individual timesteps, \AREPO{} employs a base-2 timebin hierarchy. The total simulation time is divided by powers of two, generating a set of discrete timestep sizes. Each resolution element is assigned the largest of these discrete timesteps that still satisfies all of its individual criteria. As conditions evolve, a particle's timebin is updated.

For gravitational evolution, a leapfrog, or kick-drift-kick, integration scheme is used. Here, velocities are evolved for half a timestep, followed by a full-timestep update of particle positions, and then another half-timestep velocity update. This scheme is second-order accurate.

For fluid dynamics, a similar second-order strategy is adopted. Each cell's quantities are updated using the average of fluxes computed at the beginning and end of the timestep. The end-of-timestep fluxes are evaluated on an updated mesh, with fluid variables extrapolated forward in time via local gradients and Euler's equations. This approach blends the Runge–Kutta and MUSCL–Hancock methods, achieving second-order accuracy while still requiring only a single mesh construction per timestep.

\subsubsection{Wind-based Subgrid Models}
Many physical processes important to galaxy formation occur at scales far too small to be explicitly included in simulations. This has led to the development of a wide variety of so-called ``subgrid'' models which approximate these physical effects on a coarser scale. These can be separated into two broad classes -- wind-based and explicit models. We have employed simulations which use both classes, and so we will summarize each in turn. For the wind-based models, we make use of the implementation in the Illustris~TNG simulation suite \citep{2003MNRAS.339..289S,2013MNRAS.436.3031V,2014MNRAS.444.1518V,2017MNRAS.465.3291W,2018MNRAS.473.4077P}.

On small scales, the ISM is composed of a hot ambient and cold condensed phase \citep{1977ApJ...218..148M}. However, this cold phase is unresolved for simulations with mass resolution $\gtrsim10^4\,\Msun$, and so an approximate multiphase approach was developed \citep{2003MNRAS.339..289S}. Here, each resolution element has a cold and hot component, with mass exchange allowed between the two components based on the expected supernova rate. Only elements above a certain density have a distinct cold phase.

Star formation is typically accounted for by estimating the SFR of a given gas element and then probabilistically converting all or a portion of it into a star particle. Some criterion is used to mark certain gas elements as star forming (usually all cells above a density threshold), and then estimating the SFR as the mass of the cell divided by its free-fall time multiplied by an efficiency parameter which is typically $\sim0.01$--$1$. Each star particle is a simple stellar population composed of hundreds to hundreds of thousands of stars with identical ages and compositions.

It is also necessary to include some prescription for the effects of stellar feedback, a catch-all term for things like photoionization, stellar winds, and supernovae. These processes are thought to drive large-scale outflows of gas from galaxies that suppress their star formation. A common way of accounting for this effect is to probabilistically convert star-forming gas elements into ``wind particles,'' which are decoupled from the hydrodynamic scheme and allowed to propagate for a certain time or distance from their launching point. Inclusion of such a scheme suppresses the cosmic star formation rate density and brings it in line with observational expectations \citep{2003MNRAS.339..312S}.

Another important source of energy comes from AGN. As gas accretes onto a galaxy's central SMBH, it is heated and compressed, and strong magnetic fields develop. As a result, large amounts of thermal and kinetic energy can be launched from its center, a process especially efficient in halos more massive than the Milky Way's. Significant uncertainties lie in how to include this effect. TNG takes an approach by injecting a certain amount of energy as a function of a SMBH's accretion rate, which is itself estimated from the Bondi accretion rate (limited by the Eddington accretion rate). At low accretion rates (the radio mode) energy is injected in a relatively efficient kinetic mode. At high accretion rates (the quasar mode) energy is injected through a relatively inefficient thermal mode.

Another important aspect of subgrid models comes from the chemical enrichment of gas. This will be discussed in Section~\ref{intro:ssec:alpha_elem_chem}.

\subsubsection{Explicit Subgrid Models}
More recently, as higher resolutions can be achieved ($\lesssim10^4\,\Msun$), the cold phase is more resolved, allowing for a more explicit inclusion of certain feedback sources. In particular, the stellar feedback from young stars can be included in a more physically realistic manner. This approach has been adopted by the FIRE \citep{2018MNRAS.480..800H} and VINTERGATAN simulations \citep{2021MNRAS.503.5826A}. We have made use of the SMUGGLE model, which is implemented in \texttt{AREPO} \citep{2019MNRAS.489.4233M}.

In this model, stellar winds, photoionization, and radiative feedback from massive stars is included through the deposition of mass, energy, and momentum. Individual supernovae are modeled using an estimate of the total momentum they will impart -- an approach of simply injecting thermal energy is not sufficient because the momentum-generating Sedov-Taylor phase is not resolved, and so the energy is quickly radiated away.

In this model, no hydrodynamically-decoupled wind particles are used, and there is no subgrid multiphase method. Otherwise, SMUGGLE adopts much of same structure and content as older wind-based models.

\subsubsection{Initial Conditions}
In this work, we make use of idealized, non-cosmological simulations. Unlike in cosmological simulations where the generation of initial conditions (ICs) is relatively straightforward, for idealized systems much effort must be made in generating useful ICs. In Chapters~\ref{ch:gasbar} and~\ref{ch:GSEgas} we will discuss in detail novel developments of prior routines for generating ICs. Here, we will summarize the base technique introduced in \citet{2005MNRAS.361..776S} (and based on \citet{1993ApJS...86..389H,2000MNRAS.312..859S}).

The dark matter halo and stellar bulge are constructed by sampling from a \citet{1990ApJ...356..359H} profile. The stellar disk is constructed using an exponential radial profile and isothermal vertical profile, with the radial scale length chosen to give the stellar disk a given total angular momentum. The vertical scale length of the stellar disk is a chosen fraction of the radial scale length. The gas disk is given the same radial profile while the vertical profile is solved for numerically to satisfy hydrostatic balance.

For particle velocities, the distribution function is approximated as a Gaussian. For the bulge and halo, it is assumed that the distribution function is a function only of $E$ and $L_z$. Mixed second order moments and first order in $R$ and $z$ are assumed to vanish. We can then compute $\left<v_{R}^2\right>$, $\left<v_{\phi}^2\right>$, and $\left<v_{z}^2\right>$ from the Jeans equations. For the bulge, it is assumed that $\left<v_{\phi}\right>=0$, while for the halo a small amount of rotation is added to match the angular momentum of the disk. For the latter, it is added as a fixed fraction of the circular velocity.

In the disk, the distribution function is more complicated and does not depend only on $E$ and $L_z$. $\left<v_{z}^2\right>$ is computed from the Jeans equation and it is assumed that $\left<v_{R}^2\right> = f_R \left<v_{z}^2\right>$, where $f_R\sim1$. The epicyclic approximation is assumed and so the azimuthal velocity dispersion can be inferred from the radial velocity dispersion and the force field.

For the gas, the azimuthal velocity is set to the circular velocity after accounting for the pressure support (the gas is assumed to be isothermal at $10^4\,\textrm{K}$). Its vertical and radial velocity are zero. We will build on this basic routine in Chapters~\ref{ch:gasbar} and~\ref{ch:GSEgas}.

\section{The Bar}
About two-thirds of galaxies host a bar structure at their center \citep{2000AJ....119..536E, 2007ApJ...657..790M}, as does the Milky Way \citep{1991ApJ...379..631B}. Bars exert a large influence on the internal structure and dynamics of their host galaxy. They also lead to evolution on timescales longer than the galaxy's dynamical timescale (secular evolution).

\subsection{Bar Formation}
The formation of galactic bars is a natural consequence of gravitational instability in stellar disks. Isolated stellar disks without a strong spherical potential from a bulge or dark matter halo readily develop bars. However, such simulations predict that nearly all galaxies should host bars \citep{1971ApJ...168..343H}. Hot, centrally concentrated mass distributions, such as stellar bulges or dark matter halos, stabilize disks against bar formation \citep[e.g.,][]{1973ApJ...186..467O, 1976AJ.....81...30H}.

These instabilities are driven by the amplification of non-axisymmetric disturbances through a process known as swing amplification. This process begins when shearing forces transform density waves -— initially seeded by, e.g., turbulence or tidal interactions -- from leading to trailing. During this swing, the relative motion between stars and the wave decreases, allowing self-gravity to enhance the wave's density \citep{1965MNRAS.130..125G,1966ApJ...146..810J,1981seng.proc..111T}. As these waves propagate to the galaxy's center, they transition back to leading, resulting in an amplification cycle. This manifests as an exponential growth in the second Fourier mode of the surface density, described by
\begin{equation}
\frac{\Sigma\left(R, \phi\right)}{\Sigma\left(R\right)} = \sum_{m=0}^{\infty} A_{m}\left(R\right) e^{im\left[\phi-\phi_m(R)\right]}\textrm{.}
\end{equation}
The bar structure stabilizes in the non-linear regime when stars become trapped on resonant bar-supporting orbits, typically when the $m=2$ amplitude reaches $A_2/A_0\gtrsim0.1$ \citep[e.g.][]{2018MNRAS.477.1451F,2023ApJ...947...80B}.

While internal dynamics represent the primary path to bar formation, external forces can also trigger bar development. Specifically, tidal interactions with satellite galaxies can induce bar formation even in otherwise stable disks \citep{1986A&A...166...75B}. However, this mechanism seems to be only efficient when the satellite is on a prograde orbit so that the small relative motion of the satellite and disk stars maximizes the gravitational interaction \citep[e.g.][]{2018ApJ...857....6L}. In the case of the Milky Way, there is no compelling evidence for such an interaction in its history, suggesting its bar formed through internal processes.

\subsection{Slowdown of Bars}
A key parameter of a bar's evolution is the angular rate at which its $m=2$ pattern rotates (its pattern speed $\Omega_p$). This is typically normalized by the bar's corotation radius (the radius at which the circular velocity matches the pattern speed) and the bar length defined by the parameter $\Rot=\RCR/\Rb$. Bars with $\Rot < 1.4$ are considered ``fast rotators'' while those with $\Rot > 1.4$ are ``slow rotators'' \citep{2000ApJ...543..704D}. Bars cannot maintain stable configurations with $\Rot < 1$ since orbits that extend beyond corotation are unstable \citep{1980A&A....81..198C}.

Observationally, bar pattern speeds can be measured using the Tremaine-Weinberg method \citep{1984ApJ...282L...5T}, which has been applied to large samples of galaxies through integral field unit surveys like MaNGA and CALIFA \citep{2019MNRAS.482.1733G, 2015A&A...576A.102A}, though this is a challenging measurement to reliably make due to, e.g., the difficulty of precisely estimating the position angle of the bar \citep{2019ApJ...884...23Z}. These studies consistently find that most bars are fast rotators, with $1 < \Rot < 1.4$.

This observational picture poses a challenge for theoretical models. Simulations consistently show that bars should slow down due to a transfer of angular momentum to their dark matter halos \citep{1981A&A....96..164C, 2000ApJ...543..704D}. This process, studied in detail by \citet{1984MNRAS.209..729T} and \citet{1985MNRAS.213..451W}, is analogous to the classical dynamical friction that causes satellite galaxies to spiral inward, but acts on non-axisymmetric disturbances like bars \citep{1972MNRAS.157....1L}. In the case of a bar, there is an interaction in which the bar torques halo material on near-resonant orbits, causing the bar to slow down over time. As a result, simulations tend to predict that \Rot{} increases beyond $1.4$ within a Hubble time \citep{2000ApJ...543..704D}.

This theoretical expectation -- that bars should progressively become slow rotators ($\Rot > 1.4$) over time -- is in stark contrast to the observed predominance of fast bars. This discrepancy remains one of the key outstanding problems in bar dynamics and suggests the need for some drastic solutions, like modified gravity \citep{2021MNRAS.508..926R} or altering the baryon to halo mass fraction \citep{2021A&A...650L..16F}.

\subsection{Influence of Bar on Gas}
While the interaction between bars and dark matter halos is well understood, the relationship between bars and gas disks is unsettled. Gas can exchange angular momentum with the bar through non-resonant interactions due to its collisional nature (as opposed to resonant interactions for stars), allowing it to have an outsized influence despite typically contributing only $\sim10$--$20\%$ of a galaxy's mass at the present day. Some argue that gas should enhance the bar's slowdown \citep{2003MNRAS.341.1179A}, while others suggest that the bar-driven inward flow of gas leads to an acceleration of the bar \citep{2013MNRAS.429.1949A, 2014MNRAS.438L..81A}.

This strong inflow of gas can build up central mass concentrations and nuclear stellar disks \citep{2010ApJ...719.1470V} and potentially fuel active galactic nuclei \citep{1989Natur.338...45S}. By age dating stars in the Milky Way's nuclear stellar disk, it has been shown that the Milky Way's bar formed $\sim8\,\Gyr$ ago \citep{2024MNRAS.530.2972S}. The inflow process is the natural result of mild shocks generated at the tips of the bar where gas orbits intersect \citep{1992MNRAS.259..345A,2011MNRAS.415.1027H,2013MNRAS.429.1949A}. Inside corotation, gas loses angular momentum and flows inward, whereas outside corotation, gas gains angular momentum and moves outward. Since significantly more gas is inside corotation and the effect is stronger closer to the bar, the net effect is for the gas disk to experience a bulk negative torque.

The gas disk simultaneously exerts a positive torque on the bar, competing with the negative torque exerted by the dark matter halo. We explore the possibility that the bar--gas interaction could solve the slowdown tension in Chapter~\ref{ch:gasbar}.

\section{Chemical Decomposition of the Disk}
As discussed in Section~\ref{intro:ssec:struct_prop}, much interest has been devoted towards decomposing the Galaxy into distinct components. The intention here is that each distinct component has a unique formation history or channel, and so a refined decomposition will untangle the Galaxy's history. Chapters~\ref{ch:GSEgas} and~\ref{ch:Mgdec} are concerned with the chemical decomposition of the disk into the high- and low-$\alpha$ sequences. To provide context, we first discuss the kinematic decomposition of the disk into its thin and thick components.

\subsection{Kinematic Decomposition}\label{intro:kin_decomp}
It was first realized by \citet{1983MNRAS.202.1025G} that the vertical distribution of stars near the Sun is well-fit by a double exponential. This naturally led to a decomposition of the disk into a thick and thin component. Subsequent work showed that, relative to the thin disk, the thick disk is metal-poor and $\alpha$-enhanced \citep{1998A&A...338..161F}, kinematically hotter \citep{2000AJ....119.2843C}, and old \citep{2005A&A...433..185B}. The commonly accepted formation scenario for the thick disk is that the 

These abundances can be used to decompose the Milky Way's disk, which has a long history dating back to the work of \citet{1983MNRAS.202.1025G}, who noted that the vertical distribution of stellar altitudes is well-fit by a double exponential. This led naturally to a ``thin'' and ``thick'' disk, whose membership can be reasonably determined through kinematics \citep[e.g.][]{2003A&A...410..527B}. It was quickly realized that the thick disk is more $\alpha$-enhanced than the thin disk \citep{1996ASPC...92..307G,1998A&A...338..161F}. However, it has been shown that when stars are binned in \FeH{} and \alphaFe{}, the double-exponential fit from \citet{1983MNRAS.202.1025G} vanishes, and the distribution is well-fit by a single exponential where the scalelength is \FeH{}- and \alphaFe{}-dependent \citep{2012ApJ...751..131B}.

\subsection{\texorpdfstring{$\alpha$}{α}-elements and Nucleosynthesis}\label{intro:ssec:alpha_elem_chem}
Most elements heavier than hydrogen are produced through nuclear fusion in compact object such as supernovae, dying low mass stars, and neutron star-neutron star mergers \citep[e.g.][]{2023A&ARv..31....1A}. Stars inherit the composition of the gas they form from, and are therefore records of the Galaxy's gas-phase composition. 

\subsection{\texorpdfstring{$\alpha$}{α}-elements and Nucleosynthesis}
A more promising avenue of decomposing the disk lies in the chemical composition of stars. Most elements heavier than hydrogen are produced through fusion, either in stellar interiors, supernovae, or neutron star mergers \citep[e.g.][]{2023A&ARv..31....1A}. Because stellar surface abundances generally remain unchanged throughout a star's lifetime, and stars inherit the composition of their natal gas clouds, we can use stellar chemistry to reconstruct the Galaxy's enrichment history.

While stellar abundances span a high-dimensional space of elements \citep[e.g., 32 elements in][]{2024ApJ...961L..41J}, much of the information content is contained in just two quantities: the iron abundance \FeH{} and the abundance of $\alpha$-elements relative to iron \alphaFe{}.\footnote{$\alpha$-elements are elements produced through the $\alpha$-capture process and include O, Ne, Mg, Si, and S.} Iron is produced in both Type~Ia and Type~II supernovae (SNe), making it a useful proxy for total metallicity. In contrast, $\alpha$-elements are primarily produced in Type~II SNe. These SNe channels have largely disparate timescales. Type~II SNe result from the core collapse of high-mass stars, which occurs on timescales $\lesssim40\,\Myr$, while Type Ia~SNe originate from thermonuclear explosions of white dwarfs, either through accretion-induced collapse or mergers, occurring on $\sim\Gyr$ timescales. As a result, the ratio \alphaFe{} traces the relative contribution of the two enrichment channels, and generally decreases with time as Type~Ia SNe become more important \citep{1979ApJ...229.1046T}. Because of this much attention has been aimed at the \alphaFe{}-\FeH{} plane in order to understand galactic chemical evolution and interpret the Milky Way's structure.

\subsection{High- and Low-\texorpdfstring{$\alpha$}{α} Sequences}
Recent studies have shown that the Galactic disk exhibits two chemically distinct sequences -- high- and low-$\alpha$ -- which can be separated independently of kinematics \citep{2011A&A...535L..11A,2012A&A...545A..32A}.\footnote{\citet{2003A&A...410..527B} first noted that the kinematically-selected thin and thick disks do not overlap in the \alphaFe{}-\FeH{} plane, hinting at a bimodal $\alpha$-element distribution.} The high-$\alpha$ sequence is older, more centrally compact, and more vertically extended than the low-$\alpha$ sequence \citep{2013A&A...560A.109H, 2024IAUS..377..115N}. While the thick disk is systematically more $\alpha$-enhanced than the thin disk, and there is substantial overlap between the high-$\alpha$ and thick disk populations (as well as the low-$\alpha$ and thin disk populations), it remains unclear whether the chemical and kinematic distinctions originate from the same physical process. These questions are explored in detail in Chapters~\ref{ch:GSEgas} and~\ref{ch:Mgdec}, where we examine a wide range of formation scenarios proposed in the literature and introduce two new models.

\section{Thesis Outline}
\label{sec:thesis_outline}
This thesis primarily investigates the Milky Way's bar and chemical abundance patterns, exploring their formation, evolution, and observational signatures. We approach these topics through a combination of numerical simulations and theoretical modeling, examining how the Milky Way's bar has evolved and deciphering the history encoded in the Galaxy's abundance plane.

\subsection*{Chapter 2: Stellar Bars in Isolated Gas-Rich Spiral Galaxies Do Not Slow Down}
Using simulations of Milky Way-like galaxies, we examine how gas affects the evolution of galactic bars. While theoretical models predict that bars should slow down over time due to angular momentum transfer to the dark matter halo, we find that even a small gas fraction ($\sim5\%$) can maintain a constant bar rotation rate. Our results challenge previous expectations and provide a mechanism for sustaining rapid bar rotation in galaxies.

\subsection*{Chapter 3: A Metallicity-Dependent Star Formation Gap Splits the Milky Way's $\alpha$-Sequences}
We propose that the bimodal structure in the \alphaFe{}-\FeH{} plane arises from a short-lived ($\sim300\,\Myr$) pause in star formation at fixed metallicity. Using simulations, we demonstrate that the GSE merger could have induced such a pause in the Milky Way, leading to the observed separation between high- and low-$\alpha$ sequences. This model predicts a corresponding stellar age gap at fixed metallicity.

\subsection*{Chapter 4: The Bar-driven Abundance Bimodality of the Milky Way}
Using the TNG50 cosmological simulation, we identify a galaxy that develops a chemical bimodality through the formation of its bar. This event leads to a sequence of starburst, quiescence, and rejuvenation, inducing a bimodality in the same manner as in Chapter~\ref{ch:GSEgas}, but without relying on a merger. We predict that bimodal abundance patterns should be more common in barred galaxies.