The Milky Way is a barred spiral galaxy, serving as our cosmic home and laboratory for understanding galactic evolution. With an estimated stellar mass of $\sim 5 \times 10^{10} M_{\odot}$ \citep{2015ApJ...806...96L}, it consists of several distinct components: the thin disk, thick disk, bulge, bar, and stellar halo. The disk, which contains most of the visible matter, extends to a radius of approximately 15 kpc \citep{2016ARA&A..54..529B}. 

The thin disk, with a scale height of about 300 pc, is the site of ongoing star formation and contains most of the Galaxy's gas and dust. The thick disk, discovered by \citet{1983MNRAS.202.1025G}, has a larger scale height of $\sim$ 900 pc and is composed of older, more metal-poor stars \citep{2008ApJ...673..864J}. The bulge, a dense central region of the Galaxy, harbors a mix of old and metal-rich stars, while the bar, a elongated structure of stars, extends through the bulge region \citep{2016ARA&A..54..529B}.

Surrounding these components is the stellar halo, a sparse population of old, metal-poor stars extending to distances greater than 100 kpc from the Galactic center \citep{2020ARA&A..58..205H}. This halo bears evidence of the Milky Way's accretion history, including the remnants of past merger events such as the Gaia-Enceladus-Sausage (GSE) \citep{2018Natur.563...85H, 2018MNRAS.478..611B}.

The Milky Way is embedded in a dark matter halo, which dominates the total mass of the Galaxy. Estimates of the total mass of the Milky Way, including this dark matter component, range from $0.9 - 1.6 \times 10^{12} M_{\odot}$ \citep{2019ApJ...873..118W}.

% \subsection{Overview of the bar, dynamics, and secular evolution}

The Galactic bar is a key structural component of the Milky Way, influencing its dynamics and evolution. First definitively detected by \citet{1991ApJ...379..631B}, the bar is an elongated structure of stars extending from the Galactic center. Recent studies estimate its half-length to be 5.0 ± 0.2 kpc \citep{2019MNRAS.490.4740B}.

Bars are common features in disk galaxies, with approximately two-thirds of spiral galaxies hosting such structures \citep{2000AJ....119..536E, 2007ApJ...657..790M}. They form naturally in simulations of isolated disk galaxies \citep{1971ApJ...168..343H}, although their formation can be inhibited by hot, centrally concentrated mass distributions such as bulges or dark matter halos \citep{1973ApJ...186..467O, 1976AJ.....81...30H}.

The dynamics of galactic bars are characterized by their pattern speed, typically measured using the parameter $\mathcal{R} = R_{CR}/R_b$, where $R_{CR}$ is the corotation radius and $R_b$ is the bar length \citep{2000ApJ...543..704D}. Bars with $\mathcal{R} < 1.4$ are considered "fast rotators," while those with $\mathcal{R} > 1.4$ are "slow rotators." Observational studies indicate that most extragalactic bars are fast rotators \citep{2011MSAIS..18...23C, 2015AA...576A.102A, 2019MNRAS.482.1733G, 2020MNRAS.491.3655G}.

However, theoretical models predict that bars should slow down over time due to dynamical friction with the dark matter halo \citep{1981AA....96..164C, 1992ApJ...400...80H, 2000ApJ...543..704D}. This discrepancy between observations and theory remains an open question in galactic dynamics.

Bars play a crucial role in the secular evolution of galaxies. They can drive gas inflows, potentially fueling central starbursts or active galactic nuclei \citep{1989Natur.338...45S, 1990Natur.345..679S}. They also redistribute angular momentum within the disk, influencing the radial migration of stars and the reshaping of the disk's structure \citep{2002MNRAS.336..785A, 2011A&A...527A.147M}.

Recent observations have shown that massive bars exist in disk galaxies as early as redshift z > 2 \citep{2022arXiv221008658G}, when the universe was only 2.5 billion years old. This suggests that bars play a role in galaxy evolution from very early times, even in environments with high gas fractions \citep{2020ARAA..58..157T}.

% \subsection{Overview of chemistry and the alpha-abundance bimodality}

The chemical composition of stars provides crucial insights into the formation and evolution history of the Milky Way. Elements heavier than hydrogen are produced through various nucleosynthetic processes in stars and supernovae \citep{2023A&ARv..31....1A}. The surface abundances of most stars remain largely unchanged throughout their lifetimes, serving as fossil records of the gas from which they formed.

Of particular interest in Galactic archaeology are the iron-peak elements (primarily produced in both Type Ia and Type II supernovae) and the $\alpha$-elements (primarily produced in Type II supernovae). The ratio of $\alpha$-elements to iron ([$\alpha$/Fe]) in a star reflects the relative contributions of these different types of supernovae to the enrichment of its natal gas cloud \citep{1979ApJ...229.1046T, 1986A&A...154..279M}.

A key discovery in the chemical structure of the Milky Way is the bimodality in the [$\alpha$/Fe]-[Fe/H] plane, first clearly demonstrated by \citet{2011A&A...535L..11A} and \citet{2012A&A...545A..32A}. This bimodality reveals two distinct sequences: a high-$\alpha$ sequence and a low-$\alpha$ sequence. The high-$\alpha$ sequence is generally older, more centrally concentrated, and more vertically extended than the low-$\alpha$ sequence \citep{2013A&A...560A.109H, 2024IAUS..377..115N}.

Several scenarios have been proposed to explain this bimodality:

1. Two-phase gas infall models \citep{1997ApJ...477..765C, 2009IAUS..254..191C, 2017MNRAS.472.3637G, 2019A&A...623A..60S}, where the thick disk forms rapidly from an initial gas infall, followed by a more gradual formation of the thin disk from a second infall of pristine gas.

2. Stellar migration models \citep{2009MNRAS.396..203S, 2021MNRAS.507.5882S, 2023MNRAS.523.3791C}, which propose that radial migration of stars from the inner disk creates the high-$\alpha$ sequence throughout the disk.

3. Clump formation models \citep{2019MNRAS.484.3476C, 2020MNRAS.492.4716B, 2021MNRAS.502..260B, 2023ApJ...953..128G}, suggesting that instabilities in gas-rich disks at high redshift form clumps that self-enrich to create the high-$\alpha$ sequence.

4. Merger scenarios \citep{2004ApJ...612..894B, 2005ApJ...630..298B, 2007ApJ...658...60B, 2010MNRAS.402.1489R}, proposing that gas-rich mergers enhance star formation and lead to $\alpha$-enhancement of the thick disk.

5. Feedback-driven models \citep{2021MNRAS.501.5176K}, suggesting that stellar feedback drives two phases of gas infall, creating the two sequences.

The existence of the Gaia-Sausage-Enceladus (GSE) merger event \citep{2018MNRAS.478..611B, 2018Natur.563...85H, 2020ApJ...901...48N}, estimated to have occurred 8-10 Gyr ago, has lent support to merger-based explanations. However, the exact role of this merger in shaping the $\alpha$-abundance bimodality remains a topic of active research.

% \subsection{Brief discussion of stellar ages}

Stellar ages are crucial for understanding the formation and evolution of the Milky Way. However, determining accurate ages for individual stars remains challenging, particularly for older populations \citep{2010ARA&A..48..581S}.

Various methods are used to estimate stellar ages, including:

1. Isochrone fitting for stars in clusters or binaries \citep{2017ApJ...837..162C}.

2. Asteroseismology for stars with observed pulsations \citep{2013ARA&A..51..353C}.

3. Gyrochronology, using the rotation periods of stars \citep{2007ApJ...669.1167B}.

4. Chemical clock methods, utilizing abundance ratios that evolve predictably with time \citep{2016A&A...593A..65J}.

5. Age-metallicity relations, though these are complicated by radial migration \citep{2013A&A...558A...9M}.

Recent large-scale surveys like APOGEE \citep{2017AJ....154...94M} and Gaia \citep{2016A&A...595A...1G} have provided a wealth of data on stellar parameters, including age estimates for large populations of stars. These data have revealed important trends in the age structure of the Milky Way:

1. The high-$\alpha$ sequence is generally older than the low-$\alpha$ sequence \citep{2013A&A...560A.109H}.

2. There is a vertical age gradient in the disk, with older stars typically having larger scale heights \citep{2016ARA&A..54..529B}.

3. The age-metallicity relation is relatively flat for thin disk stars, but shows a steeper trend for thick disk stars \citep{2019A&A...623A..60S}.

4. There is evidence for inside-out formation of the disk, with the inner regions being on average older than the outer regions \citep{2015ApJ...808..132H}.

Understanding the age structure of different stellar populations is crucial for constraining models of Galactic formation and evolution, including those aiming to explain the $\alpha$-abundance bimodality.

% \subsection{Thesis outline}

This thesis is structured as follows:

Chapter 1: Introduction (current chapter)

Chapter 2: Stellar Bars in Isolated Gas-Rich Spiral Galaxies Do Not Slow Down

This chapter explores the dynamics of stellar bars in gas-rich spiral galaxies using high-resolution simulations. It demonstrates that the presence of gas can arrest the process by which dark matter halos brake bars, leading to stable bar pattern speeds even in Milky Way-like disks with gas fractions as low as 5%. This work contributes to our understanding of bar evolution in realistic galactic environments and helps reconcile theoretical predictions with observational evidence of fast-rotating bars.

Chapter 3: Rising from the Ashes: A Metallicity-Dependent Star Formation Gap Splits the Milky Way's $\alpha$-Sequences

This chapter proposes a new mechanism for the formation of the $\alpha$-abundance bimodality in the Milky Way. It demonstrates, through idealized simulations mimicking the merger between the Milky Way and the Gaia-Sausage-Enceladus satellite, that a brief (~300 Myr) interruption in star formation at specific metallicities can lead to the observed bimodality. This work provides a novel perspective on the origin of the Milky Way's chemical structure and its connection to the Galaxy's merger history.

Chapter 4: Rising from the Ashes II: The Bar-driven Abundance Bimodality of the Milky Way

This chapter extends the findings of Chapter 3 by applying the proposed mechanism to a galaxy from the Illustris TNG50 cosmological simulation. It demonstrates that the $\alpha$-abundance bimodality can arise from a brief quiescent period triggered by bar-induced AGN activity, rather than a merger event. This work provides further evidence for the versatility of the proposed mechanism and its applicability in cosmological contexts.

Chapter 5: Conclusions and Future Work

This final chapter will synthesize the main findings of the thesis, discuss their implications for our understanding of galactic evolution, and propose directions for future research.