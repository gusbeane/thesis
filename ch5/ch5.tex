% !TEX root = ../ms.tex

\begin{adjustwidth}{.8cm}{0cm}
\textit{I therefore, if you are a person of the same sort as myself, should be glad to continue questioning you: if not, I can let it drop. Of what sort am I? One of those who would be glad to be refuted if I say anything untrue, and glad to refute anyone else who might speak untruly; but just as glad, mind you, to be refuted as to refute, since I regard the former as the greater benefit, in proportion as it is a greater benefit for oneself to be delivered from the greatest evil than to deliver some one else. For I consider that [one] cannot suffer any evil so great as a false opinion on the subjects of our actual argument.}

\hspace{9cm} -- Socrates
\end{adjustwidth}

\hfill \break
\noindent
This thesis has explored two key observational facts about the Milky Way. First, its bar is both old and fast-rotating. Second, at low metallicities there is a bimodal distribution in the stellar surface abundance of $\alpha$-elements. We examined the physical mechanisms behind these features via idealized numerical simulations.

In Chapter~\ref{ch:gasbar}, we proposed a solution to why the Milky Way's bar is both old and rotating rapidly. Although the bar should, in principle, lose angular momentum to the dark matter halo through resonant orbital interactions (thus slowing its pattern speed), the gas phase of the galaxy exerts a positive torque that arrests the dynamical friction process responsible for the halo's negative torque. In our simulations, this interplay effectively ``locks'' the bar at nearly constant speed over many Gyr. Our result explains why most observed bars are fast rotators without appealing to exotic physics, and suggests that slowly rotating bars should occur in gas-poor galaxies such as lenticulars.

We then turned to the emergence of the $\alpha$-bimodality in Chapter~\ref{ch:GSEgas}. In the \alphaFe--\FeH{} plane, the Milky Way separates into high- and low-$\alpha$ sequences. While there are many proposed explanations for this structure, including the sudden infusion of gas by the Gaia-Sausage-Enceladus merger, we presented a new mechanism where a short-lived ($\sim300\,\Myr$) cessation of star formation at a particular metallicity naturally produces a bimodal \alphaFe{} distribution. This scenario predicts a narrow age gap for $\sim8\,\Gyr$ old stars at $\FeH \lesssim -0.2$.

Building on these ideas, we showed in Chapter~\ref{ch:Mgdec} that bar formation in a galaxy from the cosmological simulation Illustris~TNG50 can itself drive a global quenching event, splitting the high- and low-$\alpha$ populations through a similar gap in star formation. Hence, we predict that barred galaxies should show a higher prevalence of $\alpha$-bimodalities than unbarred ones.

In current research on the Milky Way, theory struggles to keep pace with the abundance and precision of observed stellar dynamics and abundances. This gap will only widen as even more precise measurements become available. SDSS-V will soon deliver millions of abundances \citep{2017AJ....154...94M}, while the proposed Maunakea Spectroscopic Explorer will measure tens to hundreds of millions abundances down to a g-band magnitude of $20$ over its lifetime \citep{2023AN....34430108S}. But what do we actually gain from measuring so many abundances? To wit, this thesis has proposed a new explanation for the abundance bimodality that was discovered in 2011 with a sample of 1112 stars \citep{2011A&A...535L..11A} using simulation techniques developed shortly after that discovery. Will measuring abundances for a factor of 10,000 more stars truly clarify the origin story of the Milky Way? In our view, we have just started on the antipasti with the primi already on the table -- untouched -- and the secondi in the waiter's hands trying to find a place to set it down.

We need high-precision stellar ages -- in particular, of the old, metal-poor stars that the Galaxy's history is written in. A productive direction to explore in this regard comes from two complementary approaches which show particular promise: differential spectral analysis and gyrochronology measurements in combination. Differential spectral analysis has achieved remarkable precision for solar twins, reaching age uncertainties of $\sim400\,\Myr$ \citep{2018MNRAS.474.2580S}, but extending this technique to metal-poor stars has never been explored. Combining spectroscopic parameters with rotation periods from stellar light curves has been shown to improve age precision by up to a factor of 3, though the effectiveness varies across different stellar populations \citep{2019AJ....158..173A}. We plan to explore all these considerations. Nonetheless, high-precision stellar ages would truly unlock the Milky Way's history, but at the moment no one has the key.

Many years remain in this golden age of Milky Way science.